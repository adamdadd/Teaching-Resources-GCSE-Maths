%%%%%%%%%%%%%%%%%%%%%%%%%%%%%%%%%%%%%%%%%%%%%%%%%%%%%%%%%%%%%%%
%
% Welcome to writeLaTeX --- just edit your LaTeX on the left,
% and we'll compile it for you on the right. If you give 
% someone the link to this page, they can edit at the same
% time. See the help menu above for more info. Enjoy!
%
%%%%%%%%%%%%%%%%%%%%%%%%%%%%%%%%%%%%%%%%%%%%%%%%%%%%%%%%%%%%%%%

% --------------------------------------------------------------
% This is all preamble stuff that you don't have to worry about.
% Head down to where it says "Start here"
% --------------------------------------------------------------
 
\documentclass[12pt]{article}
 
\usepackage[margin=1in]{geometry} 
\usepackage{amsmath,amsthm,amssymb}
 
\newcommand{\N}{\mathbb{N}}
\newcommand{\Z}{\mathbb{Z}}
 
\newenvironment{theorem}[2][Theorem]{\begin{trivlist}
\item[\hskip \labelsep {\bfseries #1}\hskip \labelsep {\bfseries #2.}]}{\end{trivlist}}
\newenvironment{lemma}[2][Lemma]{\begin{trivlist}
\item[\hskip \labelsep {\bfseries #1}\hskip \labelsep {\bfseries #2.}]}{\end{trivlist}}
\newenvironment{exercise}[2][Exercise]{\begin{trivlist}
\item[\hskip \labelsep {\bfseries #1}\hskip \labelsep {\bfseries #2.}]}{\end{trivlist}}
\newenvironment{problem}[2][Problem]{\begin{trivlist}
\item[\hskip \labelsep {\bfseries #1}\hskip \labelsep {\bfseries #2.}]}{\end{trivlist}}
\newenvironment{question}[2][Question]{\begin{trivlist}
\item[\hskip \labelsep {\bfseries #1}\hskip \labelsep {\bfseries #2.}]}{\end{trivlist}}
\newenvironment{corollary}[2][Corollary]{\begin{trivlist}
\item[\hskip \labelsep {\bfseries #1}\hskip \labelsep {\bfseries #2.}]}{\end{trivlist}}

\newenvironment{solution}{\begin{proof}[Solution]}{\end{proof}}
 
\begin{document}
 
% --------------------------------------------------------------
%                         Start here
% --------------------------------------------------------------
 
\title{Simultaneous Equations Question}%replace X with the appropriate number
\author{Adam Dad\\ %replace with your name
Secondary Maths Tuition} %if necessary, replace with your course title
 
\maketitle
 
\question E Solve. \[\begin{split}3x + 2y = 8 \\ 2x - 5y = 18 \end{split}\]

 
\begin{solution} %You can also use solution in place of proof.
The above is an example of simultaneous equations. Simultaneous equations are required when you have 2 or more unknown variables. In the above equations these are $x$ and $y$. \\
Firstly an equations cannot be solved when there are 2 unknowns, this is why we use a second equation we can use to remove one of the unknowns at a time. \\
We can do this a few different ways:
\begin{enumerate}
    \item ADDITION. \\
        Multiply through one or both equations so that when they are added only one unknown variable remains. \\
    \item SUBTRACTION. \\
        Same as above but this time subtract rather than add. \\
        Remember minus signs!: \\
        $Positive + Positive = Positive$ \\
        $Negative + Negative = Negative$ \\
        $Positive + Negative = Negative$ \\       
    \item SUBSTITUTION. \\
        By rearranging one of the equations to $x =$ or $y =$ and then substituting this new equation into the other equation that was not rearranged leaves only one unknown. \\
\end{enumerate}

Finally after one of the  unknowns is solved this can then be substituted into one of the equations given in the question to solve for the other variable.
%Note 1: The * tells LaTeX not to number the lines.  If you remove the *, be sure to remove it below, too.
%Note 2: Inside the align environment, you do not want to use $-signs.  The reason for this is that this is already a math environment. This is why we have to include \text{} around any text inside the align environment.
\clearpage

\begin{enumerate}
    \item ADDITION. \\
    \begin{align*}
    3x + 2y &= 8 \\
    \implies 15x + 10y &= 40
    
    \end{align*}
    \item SUBTRACTION. \\
     
    \item SUBSTITUTION. \\
 
\end{enumerate}

\end{solution}
 
 
% --------------------------------------------------------------
%     You don't have to mess with anything below this line.
% --------------------------------------------------------------
 
\end{document}
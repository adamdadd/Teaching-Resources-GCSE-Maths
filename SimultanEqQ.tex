%%%%%%%%%%%%%%%%%%%%%%%%%%%%%%%%%%%%%%%%%%%%%%%%%%%%%%%%%%%%%%%
%
% Welcome to writeLaTeX --- just edit your LaTeX on the left,
% and we'll compile it for you on the right. If you give 
% someone the link to this page, they can edit at the same
% time. See the help menu above for more info. Enjoy!
%
%%%%%%%%%%%%%%%%%%%%%%%%%%%%%%%%%%%%%%%%%%%%%%%%%%%%%%%%%%%%%%%

% --------------------------------------------------------------
% This is all preamble stuff that you don't have to worry about.
% Head down to where it says "Start here"
% --------------------------------------------------------------
 
\documentclass[12pt]{article}
 
\usepackage[margin=1in]{geometry} 
\usepackage{amsmath,amsthm,amssymb}
 
\newcommand{\N}{\mathbb{N}}
\newcommand{\Z}{\mathbb{Z}}
 
\newenvironment{theorem}[2][Theorem]{\begin{trivlist}
\item[\hskip \labelsep {\bfseries #1}\hskip \labelsep {\bfseries #2.}]}{\end{trivlist}}
\newenvironment{lemma}[2][Lemma]{\begin{trivlist}
\item[\hskip \labelsep {\bfseries #1}\hskip \labelsep {\bfseries #2.}]}{\end{trivlist}}
\newenvironment{exercise}[2][Exercise]{\begin{trivlist}
\item[\hskip \labelsep {\bfseries #1}\hskip \labelsep {\bfseries #2.}]}{\end{trivlist}}
\newenvironment{problem}[2][Problem]{\begin{trivlist}
\item[\hskip \labelsep {\bfseries #1}\hskip \labelsep {\bfseries #2.}]}{\end{trivlist}}
\newenvironment{question}[2][Question]{\begin{trivlist}
\item[\hskip \labelsep {\bfseries #1}\hskip \labelsep {\bfseries #2.}]}{\end{trivlist}}
\newenvironment{corollary}[2][Corollary]{\begin{trivlist}
\item[\hskip \labelsep {\bfseries #1}\hskip \labelsep {\bfseries #2.}]}{\end{trivlist}}

\newenvironment{solution}{\begin{proof}[Solution]}{\end{proof}}
 
\begin{document}
 
% --------------------------------------------------------------
%                         Start here
% --------------------------------------------------------------
 
\title{Simultaneous Equations Question}%replace X with the appropriate number
\author{Adam Dad\\ %replace with your name
Secondary Maths Tuition} %if necessary, replace with your course title
 
\maketitle
 
\begin{question} 
1
Solve.
\begin{align*}
    3x + 2y &= 8 \\
    2x - 5y &= 18 
\end{align*}
\end{question}
\vspace{\stretch{1}}
Help and answers on following pages ................................................................................
\clearpage

\begin{solution} %You can also use solution in place of proof.
The above is an example of simultaneous equations. Simultaneous equations are required when you have 2 or more unknown variables. In the above equations these are $x$ and $y$. \\
Firstly an equations cannot be solved when there are 2 unknowns, this is why we use a second equation we can use to remove one of the unknowns at a time. \\
We can do this a few different ways:
\begin{enumerate}
    \item ELIMINATION (ADDITION). \\
        Multiply through one or both equations so that when they are added only one unknown variable remains. \\
    \item ELIMINATION (SUBTRACTION). \\
        Same as above but this time subtract rather than add. \\
     
    \item SUBSTITUTION. \\
        By rearranging one of the equations to $x =$ or $y =$ and then substituting this new equation into the other equation that was not rearranged leaves only one unknown. \\
\end{enumerate}

\begin{center}
    YOU ONLY NEED TO USE ONE OF THESE METHODS PER PROBLEM. \\
    (It's best to know how to do them all sometimes using one is easier than the other.)
\end{center}

Reminder: Don't forget sign rules!: \\
\begin{align*}
        + \times +  &= + \\
        - \times - &= + \\
        + \times - &= - \\
        - \times + &= -
\end{align*}

Finally after one of the  unknowns is solved this can then be substituted into one of the equations given in the question to solve for the other variable. \\

CHECK! Remember to substitute your values for $x$ and $y$ into the remaining equations to see if it makes sense, if one side doesn't equal the other go back and check your working for any mistakes.
%Note 1: The * tells LaTeX not to number the lines.  If you remove the *, be sure to remove it below, too.
%Note 2: Inside the align environment, you do not want to use $-signs.  The reason for this is that this is already a math environment. This is why we have to include \text{} around any text inside the align environment.
\clearpage

\begin{enumerate}
    \item ELIMINATION (ADDITION). \\
    Multiply by 5:
    \begin{align*}
        3x + 2y &= 8  \\
        \implies 15x + 10y &= 40
    \end{align*}
    Multiply by 2:
    \begin{align*}
        2x - 5y &= 8 \\
        4x - 10y &= 36
    \end{align*}
    Adding the 2 resulting equations:
    \begin{align*}
        15x + 10y &= 40 \\
        + 4x - 10y &= 36 \\
        \implies 19x &= 76 \\
        \implies x &= \frac{76}{19} \\
        \implies x &= 4
    \end{align*}
    NOT DONE YET! Substitute $x=4$ into one of the given equations:
    \begin{align*}
        3x + 2y &= 8 \\
        \implies 3(4) + 2y &= 8 \\
        \implies 12 +2y &= 8 \\
        \implies 2y &= 8 - 12 \\
        \implies 2y &= -4 \\
        \implies y &= -2
    \end{align*}
    Therefore:
    \begin{align*}
        x &= 4 \\
        y &= -2
    \end{align*}
    CHECK in unused equation:
    \begin{align*}
        2x - 5y &= 18 \\
        2(4) - 5(-2) &= 18 \\
        8 - (-10) &= 18 \\
        18 &= 18
    \end{align*}
    This proves that the $x=4$ and $y = -2$ are the correct values.
\clearpage

    \item ELIMINATION (SUBTRACTION). \\
        As this is pretty much the same as above I'll use a different multiplier so you can see that you can multiply through by any number. \\
        \\
        Let's multiply by $\frac{3}{2}$ (Just to prove it works with any number): \\
        TIP: When you don't have a calculator it can be easier to leave numbers as fractions until the  end! \\
        \begin{align*}
            \frac{3}{2} \times (2x - 5y &= 18) \\
            \implies 3x - \frac{15}{2}y &= \frac{3}{2} \times 18 
        \end{align*}
        
        Since we already have a $3x$ in the first equation lets go ahead and subtract them:
        
        \begin{align*}
            3x + 2y &= 8 \\
            - (3x - \frac{15}{2}y &= \frac{3}{2} \times 18) \\
            \implies 2y + \frac{15}{2}y &= 8 - \frac{3}{2} \times 18 \\
            \implies \frac{19}{2}y &= \frac{16}{2} - \frac{54}{2} \\
                                     &= - \frac{38}{2} \\
            \implies 8y &= -19 \\
            \implies y &= \frac{-19}{2} \\
            \implies y &= -2
        \end{align*}
    You know what we're going to to do now... Yeah that's right sub in one of the equations from the question: \\
    
        \begin{align*}
            3x + 2y &= 8 \\
            \implies 3x + 2(-2) &= 8 \\
            \implies 3x + (-4) &= 8 \\
            \implies 3x - 4 &= 8 \\
            \implies 3x &= 8 + 4 \\
            \implies x &= \frac{12}{3} \\
            \implies x &= 4
        \end{align*}
    
    CHECK in other equation: \\
    We've checked these numbers in 1. ELIMINATION (ADDITION).
\clearpage

    \item SUBSTITUTION. \\
        Multiply by $\frac{1}{2}$:
        
        \begin{align*}
            \frac{1}{2} \times (2x - 5y &= 18) \\
            \implies  x - \frac{5}{2}y &= 9
        \end{align*}
        
        Rearrange for $x =$ :
        
        \begin{align*}
            \implies x = \frac{5}{2}y + 9
        \end{align*}
        
        Sub into the other equation:
        
        \begin{align*}
            3x + 2y &= 8 \\
            3\bigg(\frac{5}{2}y + 9\bigg) + 2y &= 8
        \end{align*}
        
        Simplify:
        
        \begin{align*}
            \implies \frac{15}{2}y +2y +27 &=  8 \\
            \implies \frac{15}{2}y +2y &= 8 - 27 \\
            \implies \frac{19}{2}y &= -19 \\
            \implies y &= -2
        \end{align*}
        
        Sub $y = -2$ into equation:
        
        \begin{align*}
         3x + 2y &= 8 \\
         \implies  3x + 2(-2) &= 8 \\
         \implies 3x &= 12 \\
         \implies x &= 4
        \end{align*}
        
        Do the CHECK like previous times! See 1. ELIMINATION (ADDITION).
\end{enumerate}

\begin{center}
    \bf END OF QUESTION 1
\end{center}

\end{solution}
 
% --------------------------------------------------------------
%     You don't have to mess with anything below this line.
% --------------------------------------------------------------
\end{document}